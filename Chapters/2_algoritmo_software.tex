\chapter{Planteamiento del algoritmo en software}
\section{Recopilacion de los datos}
Para la recopilacion de los datos se utilizara la libreria wfdb que se encarga de proporcionar
funciones para leer y escribir archivos de diferentes formatos que contienen señales biomédicas,
como archivos de registro de señales (por ejemplo, formato .dat), archivos de anotaciones
 (por ejemplo, formato .atr) y archivos de cabecera (por ejemplo, formato .hea).

 Los pacientes vienen identificados por un id (por ejemplo, 101) y hay 3 ficheros por paciente, 
 con extensiones .dat, .atr y .hea

Se descarga la base de datos con la funcion de la libreria de wfdb, dldatabase que recoge 
la señal del paciente y las anotaciones de los cardiologos sobre cada pico QRS.


\lstset{language=python, breaklines=true, basicstyle=\footnotesize}
\begin{lstlisting}[frame=single]
#download the database if not available
if os.path.isdir("mitdb"):
	print('You already have the data.')
else:
	wfdb.dl_database('mitdb', 'mitdb')
\end{lstlisting}

Los pacientes de la base de datos se han hecho una prueba de 30 mins lo que en la señal 
equivale a 650000 samples.

\lstset{language=python, breaklines=true, basicstyle=\footnotesize}
\begin{lstlisting}[frame=single]
sampfrom = 0
sampto = 650000
record = wfdb.rdsamp('mitdb/102', sampfrom=sampfrom, sampto=sampto)
annotation = wfdb.rdann('mitdb/102', 'atr', sampfrom=sampfrom, sampto=sampto)
\end{lstlisting}

Por ultimo, para visualizar esta señal con las anotaciones de los cardiologos y poder comparar 
con las anotaciones que realiza el algoritmo se usara la libreria matplotlib.pyplot.

Con esto se mostrara la señal original con las anotaciones y la señal filtrada con las anotaciones
del algoritmo como en \Cref{fig:102filtradoysinfiltrar}

\lstset{language=python, breaklines=true, basicstyle=\footnotesize}
\begin{lstlisting}[frame=single]
#plot the signal
#add markers to the original signal
ax[0].plot(original_signal)
ax[1].plot(filtered_signal)
ax[0].set_xlabel('Samples')
ax[0].set_ylabel('Lead I')
ax[1].set_xlabel('Samples')
ax[1].set_ylabel('Lead I')


#Making the upper signal
for pos, sym in zip(annotation.sample, annotation.symbol):
    pos -= sampfrom
    ax[0].plot(pos, original_signal[pos], 'go', markersize=4, markerfacecolor='white')
    if(sym == "A" or sym == "V" or sym == "a"):
        ax[0].text(pos+10, original_signal[pos], sym, color='red')
    else:
        ax[0].text(pos+10, original_signal[pos], sym, color='orange')
\end{lstlisting}

\section{Filtrado de la señal original}
Este filtrado es llevado a cabo por el filtrado IIR.

El filtrado IIR, que significa "Infinite Impulse Response" (respuesta infinita al impulso),
es un tipo de filtro utilizado en el procesamiento de señales digitales y analógicas.

La formula que se utilizara para el filtrado es

\[ Y[i] = \sum_{k=0}^{N_x -1} b_k \cdot x[i-k] \]

Con lo que b son los coeficientes y x la señal a filtrar

Los coeficientes son 

\lstset{language=python, breaklines=true, basicstyle=\footnotesize}
\begin{lstlisting}[frame=single]
filter_taps_99_6_28 = [  -0.022064516719960285,  -0.01248143191787449,  -0.014272402451704691,  -0.01482693239340742,  -0.013830723879265979,  -0.011121793043893711,  -0.006713818327951467,  -0.0008089345809688635,  0.006216127611731759,  0.013857892265470411,  0.02152515460626638,  0.028576965837020497,  0.034398765027779624,  0.03844530602431575,  0.04025284446627053,  0.039530808378156514,  0.0361389960690063,  0.030158595190338606,  0.021892977443216645,  0.01187877085962161,  0.0008679402497379659,  -0.010209492943778627,  -0.020330840764461847,  -0.02848685236456877,  -0.03381377331912697,  -0.03574615252534521,  -0.03413529962465352,  -0.02933049408837693,  -0.022191934029519987,  -0.014013910197915145,  -0.006355535107839268,  -0.0008277532244847638,  0.0011959567983135999,  -0.0011634462076121899,  -0.008076950035987168,  -0.018875074584752207,  -0.0320701386231862,  -0.04551445963999694,  -0.05668375253673195,  -0.0630716859149452,  -0.06260700837237966,  -0.05404472876437898,  -0.037193610758084875,  -0.013167513231520076,  0.01590495809018272,  0.04685141764647888,  0.07609319762164894,  0.10009947011806296,  0.11585338656670081,  0.12133902448458535,  0.11585338656670081,  0.10009947011806296,  0.07609319762164894,  0.04685141764647888,  0.01590495809018272,  -0.013167513231520076,  -0.037193610758084875,  -0.05404472876437898,  -0.06260700837237966,  -0.0630716859149452,  -0.05668375253673195,  -0.04551445963999694,  -0.0320701386231862,  -0.018875074584752207,  -0.008076950035987168,  -0.0011634462076121899,  0.0011959567983135999,  -0.0008277532244847638,  -0.006355535107839268,  -0.014013910197915145,  -0.022191934029519987,  -0.02933049408837693,  -0.03413529962465352,  -0.03574615252534521,  -0.03381377331912697,  -0.02848685236456877,  -0.020330840764461847,  -0.010209492943778627,  0.0008679402497379659,  0.01187877085962161,  0.021892977443216645,  0.030158595190338606,  0.0361389960690063,  0.039530808378156514,  0.04025284446627053,  0.03844530602431575,  0.034398765027779624,  0.028576965837020497,  0.02152515460626638,  0.013857892265470411,  0.006216127611731759,  -0.0008089345809688635,  -0.006713818327951467,  -0.011121793043893711,  -0.013830723879265979,  -0.01482693239340742,  -0.014272402451704691,  -0.01248143191787449,  -0.022064516719960285]

\end{lstlisting}

Para el filtrado se usa la funcion lfilter de la libreria scipy.signal

\lstset{language=python, breaklines=true, basicstyle=\footnotesize}
\begin{lstlisting}[frame=single]
    filtered_signal = lfilter(filter_taps_99_6_28, 1.0, original_signal)
\end{lstlisting}
TODO a y formula completa ademas de una mejor explicacion

\section{Detección de picos QRS}

El algoritmo de deteccion de picos esta representada en esta funcion que 
recibe la señal filtrada e intenta detectar los picos QRS.

Este algoritmo esta basado en el que se usa en el documento  https://www.mdpi.com/2079-9292/10/19/2324 
donde en el 4.1.2 muestran una maquina de estados del proceso que realizan.

\begin{figure}[h!]
    \centering
    \includegraphics[width=0.6\textwidth]{./Images/img_algoritmo/fsm_mdpi.png}
    \caption{Maquina de estados de algoritmo de deteccion de picos de estudio de caracterizacion de señales usando polinomios de Hermite}
    \label{fig:fsm_mpdi}
\end{figure}

Si bien nuestro algoritmo es distinto a ese, se replica el esperar a 72 muestras
para asegurar de que no se encuentra un pico superior y asi considerarlo como un pico QRS.

Es por ello que definimos la variable samples\_around\_peak como 72 para comparar dicha condición.

Para hallar el pico mas alto se necestia definir un pico en last\_peak y si se encuentra otro pico se produce

\lstset{language=python, breaklines=true, basicstyle=\footnotesize}
\begin{lstlisting}[frame=single]
last-peak = max(last-peak,signal[i])
\end{lstlisting}

Sin embargo hay un problema y es que cuando se detecte un pico QRS, es decir cuando se haya detectado el pico 
mas alto despues de haber pasado 72 samples se restauran los valores para empezar a detectar nuevos picos y al
haber ruido el algoritmo podria detectar falsos picos QRS asi que por ello se implementa un cutoff.

El cutoff es representado como una funcion descendente que parte de cada pico localizado y mientras no se haya
encontrado ningun pico, el valor de dicha funcion va decreciendo. La principal funcion del cutoff es evitar que
el algoritmo detecte picos con el ruido y por ello se ha ajustado para que no ocurra el problema anterior y 
ser capaz de detectar todos los picos QRS.

La funcion del cutoff es la siguiente.
\lstset{language=python, breaklines=true, basicstyle=\footnotesize}
\begin{lstlisting}[frame=single]
def calcular_cutoff(cutoff):
    cutoff = cutoff - cutoff/(256 - 64)
    return cutoff
\end{lstlisting}

Esta funcion es llamada cuando no se ha encontrado un nuevo pico y decrementa su valor, cuando se localiza un
nuevo pico, el cutoff pasa a tener el valor del pico localizado.

\lstset{language=python, breaklines=true, basicstyle=\footnotesize}
\begin{lstlisting}[frame=single]
    def extract_peak_indices(signal, total_samples):
        samples_around_peak = total_samples // 2
        last_peak = None
        last_index = None
        peak_indices = []
        cutoff = 0
        for i in range(samples_around_peak-1, len(signal)):
            if last_peak == None:
                last_peak = signal[i]
                last_index = i
                cutoff = calcular_cutoff(cutoff)
            else:
                if signal[i] > last_peak and signal[i] > cutoff:
                    last_peak = signal[i]
                    last_index = i
                    cutoff= signal[i]
                else:
                    if (i - last_index) >= samples_around_peak and last_peak > cutoff:
                        peak_indices.append(last_index)
                        cutoff = calcular_cutoff(cutoff)
                        last_peak = None
                        last_index = None         
                    else:
                        cutoff = calcular_cutoff(cutoff)
            cutoff_plot.append(cutoff)
        ax[1].plot(range(samples_around_peak-1, len(signal)),cutoff_plot)
        return peak_indices
\end{lstlisting}

La salida de dicha funcion es un buffer de samples que sirven como indices para indicar donde se han encontrado
los picos QRS y asi poder pasar al modulo de deteccion de arritmias.

\section{Detección de arritmias}

El algoritmo de deteccion de arritmias se encarga de ver si se ha producido una arritmia segun la
distancia entre los picos.

En la deteccion de arritmias es de vital importancia establecer un límite en la distancia entre los picos
para poder considerar que ha habido una arritmia o no, esta tarea solo se pudo hacer probando con diferentes
rangos y viendo el indice de aciertos producidos en las pruebas a cada paciente de las que se hablara más adelante. 

El algoritmo va almacenando distancias entre los picos QRS (es por ello que en la primera iteración no se almacena nada)
y se declaran varias variables.

\begin{itemize}
    \item last\_distance: se utiliza para almacenar la ultima distancia recogida y asi poder compararla con la distancia 
    actual en calculos posteriores
    \item counter\_buffer: utilizado para tener el valor de la posicion del buffer donde se escribe.
    \item counter\_arrythmia: utilizado para indicar si la distancia anterior fue una arritmia.
    \item TNRange: Se utiliza para indicar si hay una distancia mas grande de lo normal entre 2 picos QRS producido
    por una arritmia. Es importante tener esta distancia en cuenta ya que si el ritmo del paciente vuelve a la
    normalidad se compararia la distancia entre el ritmo normal del paciente con el ritmo extendido por la arritmia,
    ya que de no tenerlo en cuenta el algoritmo lo clasificaria como arritmia como se puede ver en la \cref{fig:senial_explicacion_TNRANGE}, por ello se compara con un valor anterior
    que sea el ritmo normal del paciente.
\end{itemize}

\begin{figure}[h!]
    \centering
    \includegraphics[width=0.9\textwidth]{./Images/img_algoritmo/senial_explicacion_TNRANGE.png}
    \caption{Cuando se detecta una arritmia, a veces, la siguiente distancia es considerablemente mas grande de lo normal. Para no detectar falsos positivos, se omite esa distancia}
    \label{fig:senial_explicacion_TNRANGE}
\end{figure}

Por ello si se ha detectado una arritmia, la siguiente distancia se compara con
la tercera ultima distancia escrita en el buffer que posiblemente sea una distancia causada por un ritmo normal. 
Si no se da el caso, se compara con last\_distance.

la funcion que compara las distancias devuelve un char que va a ser el que se vaya a plotear en la grafica, 
si el char es "N" significa que se ha detectado un ritmo normal y por tanto solo se plotea. Sin embargo si el 
resultado es "T" significa que la distancia es mas corta de lo normal, se detecta la arritmia y se ponen
counter\_arritmia a 1 para saber que la distancia es mas corta y TNRange a true para que el algoritmo sepa 
que la distancia que venga despues puede ser una ampliada. 

\lstset{language=python, breaklines=true, basicstyle=\footnotesize}
\begin{lstlisting}[frame=single]
    #init the first distance, the first beat doesn't count
    posant = peaks[0]
    last_distance = peaks[1] - peaks[0]

    pos_buffer = [last_distance]
    counterBuffer = 0
    #this var refers to the distance left in between an arrythmia and normal rythm 
    # which tends to be longer. To avoid detection problems we will not compare this distance so the detection can be more precise
    TNRange = False
    counter_arrrythmia = 0
    for pos in peaks[1:]:
        act_distance = pos - posant
        pos_buffer.append(act_distance)
        counterBuffer += 1
    
        if(TNRange==True and counter_arrrythmia == 0):
            sym = get_frecuency_in_char(pos_buffer[counterBuffer - 3],act_distance)
    
            TNRange=False
        else:
            if(TNRange == True):
                counter_arrrythmia -= 1
            sym = get_frecuency_in_char(last_distance,act_distance)
        
    
        if(sym == "N"):
            ax[1].plot(pos, filtered_signal[pos], 'go', markersize=4, markerfacecolor='green')
            ax[1].text(pos-30, filtered_signal[pos], sym, color='green')       
        elif(sym == "T"):
            ax[1].plot(pos, filtered_signal[pos], 'go', markersize=4, markerfacecolor='red')
            ax[1].text(pos-30, filtered_signal[pos], sym, color='red')
            TNRange = True
            counter_arrrythmia = 1
        
        posant = pos
        last_distance = act_distance
\end{lstlisting}

La funcion get\_frecuency\_in\_char() se encarga de calcular las distancias entre el ritmo actual y un ritmo normal. 
Para ello recibe como entrada ambas distancias.

Para empezar se calcula el gap que es simplemente la diferencia que tiene el la distancia anterior con la actual.
Despues se calcula el porcentaje de la diferencia de distancia con la distancia anterior que sabemos que va a ser 
un ritmo normal.

Si ese porcentaje es mayor que el 15\% entonces se considera que la distancia normal es mucho mayor que la actual
y por tanto como la distancia actual entre 2 picos es pequeña, se da por hecho que hay una arrimtia.

Notese que no le damos importancia si el gap da como resultado un número negativo de cualquier tamaño, esto se debe
a que este proyecto solo esta pensado para detectar contracciones prematuras del corazon, por ende solo necesitamos 
saber si la distancia actual es menor que la anterior. Además ningun paciente parece padecer ninguna arritmia de otro
tipo.

\lstset{language=python, breaklines=true, basicstyle=\footnotesize}
\begin{lstlisting}[frame=single]
def get_frecuency_in_char(last_distance,act_distance): 

    gap = last_distance - act_distance

    percentaje = (gap / last_distance) * 100

    if(percentaje > 15):
        ret = "T"
    else:
        ret = "N"
    return ret

\end{lstlisting}

\section{Pruebas con el algoritmo}

