\chapter{Conclusión}
Este proyecto trata de buscar una solución simple para la detección de arritmias de una señal de un electrocardiograma, Para la elaboración de este proyecto, 
se ha estudiado el comportamiento de las arritmias, viendo la base de datos de MIT y estudiando el comportamiento de las arritmias anotadas se observó que la inmensa mayoría de las arritmias que ocurrían eran dadas por una contracción prematura del corazón, por tanto el proyecto, aunque inicialmente se pensó detectar el mayor tipo de arritmias posibles, al no ver ningún ejemplo claro de arritmia no producida por una contracción prematura el proyecto solo se centró en detectar dichas arritmias.

Se realizó un prototipo en \textit{python} que sirvió para crear el algoritmo y probarlo con facilidad. Este prototipo inicia
con un filtrado de la señal original aplicando el filtrado FIR. Seguidamente se aplica un algoritmo de detección de 
picos QRS sobre la señal filtrada que busca el pico más alto que además sobrepase el \textit{cutoff} dinámico establecido. Finalmente
se aplica el algoritmo de detección de arritmias calculando la distancia entre el pico actual con el anterior y comparándola con una
distancia anterior de un ritmo normal.

Para la implementación de  \textit{hardware}  se utilizaron 3 módulos principales que son el módulo de filtrado, el módulo de detección de picos 
y el módulo de detección de arritmias. Además estos módulos están contenidos en un módulo principal. Para hacer las pruebas sobre
estos módulos, se añaden 2 módulos adicionales de input de señal y output donde se comparan los resultados de las anotaciones. Además 
se evalúan los resultados mediante una simulación al crear un \textit{testbench}.

Para las pruebas en  \textit{hardware}  se utiliza la FPGA \textit{Artix 7} en la placa \textit{Basys3}, ya que es la FPGA que se usa en el estudio y aunque no sea capaz de albergar los 30 minutos de pruebas en la RAM, con menos pruebas tiene un buen desempeño.

En el fichero .xdc se ha establecido un periodo específico teniendo en cuenta la frecuencia de las muestras que es de 360 sps y da un periodo de 1852 ns. Gracias al reporte de \textit{timing} se halla que el mínimo periodo de funcionamiento es de 6,49 ns.

Según el reporte de potencia el consumo de la placa es de 0,069 vatios, y solamente el algoritmo consume menos de 0,001 vatios, lo que resulta en un consumo bajo incluso para un uso continuo de este. Comparándolo con otros proyectos similares, el consumo dinámico es menor.

\chapter*{Conclusion}

This project tries to find a simple solution for the detection of arrhythmias from an electrocardiogram signal. 
arrhythmias from an electrocardiogram signal. For the elaboration of this project, 
the behavior of the arrhythmias has been studied, looking at the database of MIT and studying the behavior of the
the behavior of the noted arrhythmias, it was observed that the vast majority of the arrhythmias that occurred 
the vast majority of the arrhythmias that occurred were due to premature contraction of the heart.
initially intended to detect as many arrhythmias as possible, but did not see any clear examples of arrhythmias not caused by premature
of arrhythmia not caused by premature contraction, the project only focused on detecting these arrhythmias.

A prototype was made in Python which was used to create the algorithm and test it easily. This prototype starts
with a filtering of the original signal by applying the FIR filtering. Next, a QRS peak detection algorithm is applied to the filtered signal. 
QRS peak detection algorithm is then applied to the filtered signal to find the highest peak that also exceeds the established dynamic cutoff.
Finally, the arrhythmia detection algorithm is applied by calculating the distance between the current peak and the previous peak and comparing it with a previous distance of a normal rhythm.
previous distance of a normal rhythm.

For the hardware implementation, 3 main modules will be used, which are the filtering module, the peak detection module and the peak detection module. 
and the arrhythmia detection module. In addition these modules are contained in a main module. In order to test on these modules
These modules are tested by adding 2 additional modules for signal input and output where the results of the annotations are compared. In addition 
the results are evaluated through a simulation by creating a testbench.

For the hardware tests, the Basys3 FPGA is used since it is the FPGA used in the study and although it is not capable of 
Although it is not able to hold the 30 minutes of tests in RAM, with less tests it has a good performance.

In the .xdc a specific period has been established taking into account the frequency of the samples which is 360 sps and gives a period of 1852 ns. Thanks to the .xdc, a period of 1852 ns has been established. 
Thanks to the timing report we found that the minimum operating frequency is 6.49 ns.

According to the power report the power consumption of the board is 0.069 W which results in a low power consumption even for a continuous use of the board.
Compared to other similar projects, the dynamic power consumption is lower.


\chapter{Trabajo futuro}

Como trabajo futuro, se ha propuesto sincronizar el proyecto con un dispositivo capaz de detectar latidos del corazón para mostrar pruebas en tiempo real y demostrar la eficacia del proyecto en cualquier persona. Además, se considera la posibilidad de implementar todo lo anterior en un dispositivo similar a un reloj para crear un producto que cumpla todos los objetivos mencionados en el proyecto.

También se ha pensado en realizar un trabajo de investigación para explorar alternativas en el ámbito médico y compararlas en términos de consumo y eficacia. Este trabajo de investigación proporcionaría información valiosa para mejorar y optimizar el proyecto actual, así como para identificar oportunidades para futuras investigaciones y desarrollos.

Por último, se ha contemplado la posibilidad de ampliar el proyecto para detectar más tipos de arritmias siguiendo los patrones que estas presentan, lo que permitiría desarrollar un algoritmo más completo y versátil para la detección de arritmias cardíacas.
