\titleformat{\chapter}[display]
{\normalfont\huge\bfseries}{Capítulo \thechapter}{0.5em}{\huge}
\titlespacing*{\chapter}{0pt}{-1.25cm}{25pt}
\chapter{Introducción}
\section{Arritmias}
Las enfermedades cardiovasculares son la primera causa de muerte en el mundo y una de las causas mas comunes
de estas enfermedades son las arritmias.

Una arritmia cardiaca es una alteración en el ritmo normal del corazón. Si se produce una arritmia, el corazón 
puede latir demasiado rápido, demasiado lento o de manera irregular. Esto puede provocar síntomas como palpitaciones,
mareos, falta de aire e incluso desmayos y estas pueden llegar a ser mortales.

Los cardiologos utilizan dispositivos como un Holter para generar tiras de ritmo o electrocardiogramas, que es un 
diagrama que representa los latidos del corazon y con eso pueden llegar a detectar arritmias.

(IMAGEN ECG)

En este proyecto se tratara de solucionar las arritmias en las que se produce una contraccion prematura del corazon
como las contracciones prematuras del corazón. Estas arritmias se 
pueden detectar con un electrocardiograma (ECG) que es un diagrama de los latidos del corazon.



\section{Algoritmo de deteccion}
Dado que para detectar arritmias correctamente se necesitan varios años de cardiologia,el algoritmo de 
deteccion que se utilizara consistira en detectar las arritmias unicamente usando los picos QRS del electrocardiograma.

Un pico QRS en un electrocardiograma es causado por la contaccion del ventriculo al bombear la sangre por las arterias.
Este es el impulso electrico mas fuerte que el corazon produce en cada latido. En este proyecto utilizaremos estos picos
para comparar la distancia entre ellos y poder ver si se ha producido una arritmia. 

(IMAGEN QRS)

\subsection{Filtrado}
Como se puede ver en las imagenes es conveniente hacer un filtrado de las tiras de ritmo para poder detectar mejor
los picos QRS. Ya que el filtrado centra la onda en el 0 y evita fallos en el algoritmo de deteccion de picos del 
que se hablará mas adelante. 

En la creacion del proyecto se ha intentado no filtrar la onda para comprobar si se obtienen mejores resultados que
sin dicho filtrado pero no se ha dado el caso por las irregularidades de la misma.

\section{Pruebas con pacientes}
Se han realizado las pruebas con unos resultados del Instituto de Tecnología de Massachusetts (MIT) en el que se han
recogido tiras de ritmo de media hora de varios pacientes con edades diversas y algunos de ellos llevan un marcapasos
que actua cuando el corazón no bombea la sangre lo suficientemente fuerte, es decir que el pico QRS no es tan prominente
y se necesita la ayuda de dicho marcapasos para proporcionar el impulso electrico necesario.

Estas pruebas han sido analizadas por cardiologos y se ha indicado donde el paciente padece una arritmia y donde el ritmo
es normal y donde se ha producido un error en la lectura de la señal. Tambien muestra informacion menos relevante como la 
activacion del marcapasos.

\section{Utilizacion de las FPGAs}
El algoritmo se implementara en una FPGA para optimizar el rendimiento del algoritmo, paralelizar 
las instrucciones, disminuir el consumo de energia ademas de la latencia a la hora de procesar la señal a tiempo real.
Ademas se ha pensado que este algoritmo se utilize en un dispositivo portable para que los usuarios puedan
tener una idea de las contracciones prematuras que tienen y tomen la decision de ir a un cardiologo.

\section{Objetivos del proyecto}
Los objetivos de este proyecto es tener una solucion para detectar contracciones prematuras a tiempo real en un largo periodo 
de tiempo.


\section{Plantilla para usos de la herramienta}
		The document is divided into \texttt{chapters}, \texttt{sections}, and \texttt{subsections}.

		Some important references are \cite{einstein,latexcompanion,knuthwebsite}.

		To add paragraphs in the document, 
		one line break is not enough,

		two line breaks are needed.

		An itemized list:

		\begin{itemize}
			\item An item.
			\item Another item.
			\item Final item.
		\end{itemize}

		An enumerated list:

		\begin{enumerate}
			\item First item.
			\item Second item.
			\item Third item.
		\end{enumerate}

		A figure with an image is presented in \Cref{fig:logo_ucm}. Note that it floats away and latex places it where convenient.

		\begin{figure}[h!]
			\centering
			\includegraphics[width=0.4\textwidth]{./Images/escudo_ucm.pdf}
			\caption{Sample figure}
			\label{fig:logo_ucm}
		\end{figure}

		Tables work in the same way, as seen in \Cref{tab:tabla}

		\begin{table}[h!]
			\centering
			\begin{tabular}{c|c|c}
				Row & English & Español \\\hline\hline
				1 & One & Uno \\
				2 & Two & Dos \\
			\end{tabular}
			\caption{Sample table}
			\label{tab:tabla}
		\end{table}
	\blindtext