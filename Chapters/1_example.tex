\titleformat{\chapter}[display]
{\normalfont\huge\bfseries}{Capítulo \thechapter}{0.5em}{\huge}
\titlespacing*{\chapter}{0pt}{-1.25cm}{25pt}
\chapter{Introducción}
Las enfermedades cardiovasculares son la primera causa de muerte en el mundo y una de las causas mas comunes
de estas enfermedades son las arritmias como las contracciones prematuras del corazón. Estas arritmias se 
pueden detectar con un electrocardiograma (ECG) que es un diagrama de los latidos del corazon.

(IMAGEN ECG)

Dado que para detectar las arritmias correctamente se necesitan varios años de cardiologia,el algoritmo de 
deteccion que se utilizara consistira en detectar las arritmias segun la distancia de los picos QRS. Con lo
que el algoritmo detectara cuales son esos picos y comparara las distancias entre ellos haciendo antes un 
filtrado sobre la señal inicial. Para intentar lograr mayor precision en la deteccion de picos se hizo el 
algoritmo sobre la señal inicial sin filtrar pero los resultados llegaron a ser peores.  

(IMAGEN QRS)

El algoritmo se implementara en una FPGA para optimizar el rendimiento del algoritmo, paralelizar 
las instrucciones, disminuir el consumo de energia ademas de la latencia a la hora de procesar la señal a tiempo real.
Ademas se ha pensado que este algoritmo se utilize en un dispositivo portable para que los usuarios puedan
tener una idea de las contracciones prematuras que tienen y tomen la decision de ir a un cardiologo.

Los objetivos de este proyecto es tener una solucion para detectar contracciones prematuras a tiempo real en un largo periodo 
de tiempo.


\section{¿Qué es APRS?}
		The document is divided into \texttt{chapters}, \texttt{sections}, and \texttt{subsections}.

		Some important references are \cite{einstein,latexcompanion,knuthwebsite}.

		To add paragraphs in the document, 
		one line break is not enough,

		two line breaks are needed.

		An itemized list:

		\begin{itemize}
			\item An item.
			\item Another item.
			\item Final item.
		\end{itemize}

		An enumerated list:

		\begin{enumerate}
			\item First item.
			\item Second item.
			\item Third item.
		\end{enumerate}

		A figure with an image is presented in \Cref{fig:logo_ucm}. Note that it floats away and latex places it where convenient.

		\begin{figure}[h!]
			\centering
			\includegraphics[width=0.4\textwidth]{./Images/escudo_ucm.pdf}
			\caption{Sample figure}
			\label{fig:logo_ucm}
		\end{figure}

		Tables work in the same way, as seen in \Cref{tab:tabla}

		\begin{table}[h!]
			\centering
			\begin{tabular}{c|c|c}
				Row & English & Español \\\hline\hline
				1 & One & Uno \\
				2 & Two & Dos \\
			\end{tabular}
			\caption{Sample table}
			\label{tab:tabla}
		\end{table}
	\blindtext