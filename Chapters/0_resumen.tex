\chapter{Resumen}

	La caracterización automática de la señal ECG es de importancia crítica 
	en el monitoreo y diagnóstico del paciente, es por ello que en este trabajo 
	lo que se pretende es detectar las arritmias de los pacientes dado un ECG 
	(electrocardiograma) en forma de una señal analizable con un algoritmo previamente prototipado y probado en software 
    que funciona a tiempo real cuyo proposito es detectar los picos QRS del paciente y detectar arritmias comparando 
    las distancias que tienen los picos actuales con la distancia que formaron los picos anteriores. Para el diseño 
    hardware se organizará el programa en distintos módulos principales con lo que cada uno desempeñara una funcion para 
    seguir los pasos del prototipo en software y realizar las tareas de filtrado de señal, deteccion de picos y deteccion de arritmias.
    Estos modulos estan contenidos en un modulo principal y a su vez se incluye en un testbench para realizar pruebas. Las pruebas se 
    realizaran metiendo señales del electrocardiograma de un paciente en una memoria para que el programa las procese, tambien se meteran 
    en otra memoria las anotaciones para comprobar si se han producido arrimtias y asi comprobar la eficacia de la replicacion del programa
    del prototipado. Para los resultados experimentales, se especificara la FPGA utilizada, la frecuencia ideal, el reporte de timing y el 
    consumo en W necesarios para el programa. 

\section{Abstract}

    The automatic characterization of the ECG signal is of critical importance in patient monitoring and diagnosis. 
	in the monitoring and diagnosis of the patient, that is why in this work the aim is to 
	the aim of this work is to detect arrhythmias in patients given an ECG (electrocardiogram) in the form of a signal that can be 
	(electrocardiogram) in the form of an analyzable signal with an algorithm previously prototyped and tested in a software 
    whose purpose is to detect the patient's QRS peaks and to detect arrhythmias by comparing the distances of the QRS 
    the distances that the current peaks have with the distance that the previous peaks formed. For the hardware 
    hardware, the program will be organized in different main modules, each of which will perform a function to follow the steps of the prototype in software. 
    The software prototype steps and performs the tasks of signal filtering, peak detection and arrhythmia detection.
    These modules are contained in a main module and in turn are included in a testbench for testing. The tests will be 
    The tests will be performed by putting electrocardiogram signals of a patient in a memory for the program to process them. 
    also put in another memory the annotations to check if they have been produced and thus check the effectiveness of the replication of the prototyping program.
    of the prototyping program. For the experimental results, we will specify the FPGA used, the ideal frequency, the timing report and the W consumption needed for the program. 
    consumption in W needed for the program. 
